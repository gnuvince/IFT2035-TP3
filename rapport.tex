\documentclass[10pt]{article}

\usepackage[utf8]{inputenc}
\usepackage[french]{babel}
\usepackage{amsmath}
\usepackage{amsfonts}
\usepackage{amssymb}
\usepackage{graphicx}
\usepackage{parskip}

\newcommand{\usage}[1]{\textbf{Utilisation: }\emph{#1}}

\begin{document}

\title{Rapport - Devoir 3}
\date{Décembre 2010}
\author{Vincent Foley-Bourgon (FOLV08078309) \and
    Eric Thivierge (THIE09016601)}

\maketitle

\section{$N$-Dames}

\subsection{Fonctionnement général}

Afin de vérifier (ou rechercher) une solution au problème des
$N$-Dames, la première étape consiste à construire une matrice
d'attaque $M$ \footnote{Une matrice est une liste des listes.}. Cette
matrice de $N \times N$ contient dans ses cellules la valeur $t$
(vrai) si une case est attaquée par une dame et $f$ (faux) si une case
est sécuritaire; initialement, $M$ ne contient que des $f$.

Des dames sont placées ligne par ligne; si on tente de placer une dame
sur une case $f$ de la matrice, cette position est valide et on passe
à la ligne suivante. Si on tente de placer une dame sur une case $t$
de la matrice, cette position est invalide, et le programme effectue
un retour-arrière et tentera une position différente.

Voici un exemple de la matrice $M$ si on place une dame à la position
$(1,2)$ (première ligne, deuxième colonne):

\[
\begin{bmatrix}
  \textbf{t} & \textbf{t} & \textbf{t} & \textbf{t} \\
  \textbf{t} & \textbf{t} & \textbf{t} & f \\
  f & \textbf{t} & f & \textbf{t} \\
  f & \textbf{t} & f & f
  \end{bmatrix}
\]

Si le programme réussit à placer $N$ dames sur chacune des lignes sans
qu'aucune dame soit attaquée par une autre dame, on a un placement
valide.

\subsection{Relations}

\subsubsection{repeter}

\usage{repeter(N, E, L)}

Cette relation mettra $N$ fois l'élément $E$ dans la liste $L$. Par
exemple, \emph{repeter(4, x, L) = [x, x, x, x]}. \emph{repeter} est
une relation récursive: répéter 0 fois donne la liste vide, sinon on
``cons'' (pour utiliser la terminologie de Scheme) $E$ au résultat de
répéter $N-1$ fois l'élément $E$.

\subsubsection{matrice}

\usage{matrice(N, E, Mat)}

Cette relation crée une matrice $N \times N$ où tous les éléments sont
initialisés à $E$. Cette relation est utilisée au début du programme
pour créer la matrice d'attaque initiale. Cette relation utilise la
relation \emph{repeter} en créant une liste de $N$ éléments et en
répétant cette liste $N$ fois.

\subsubsection{changerCol}

\usage{changerCol(I, E, L, L2)}

Cette relation change le $I$ème élément de $L$ par $E$ donnant la
nouvelle liste $L2$. Cette relation est récursive; si $I \ne 1$, on
``cons'' l'élément actuel au résultat de changer la colonne $I-1$ du
reste de la liste.


\subsubsection{changerCell}

\usage{changerCell(I, J, E, M, M2)}

Cette relation change l'élément en position $(I, J)$ de la matrice $M$
par $E$ donnant la nouvelle matrice $M2$. Cette relation est
récursive; si $I \ne 1$, on passe à la ligne suivante. Une fois
arrivé à la bonne ligne, on modifie l'élément à la colonne $J$ par
$E$.


\subsubsection{occuperCol}

\usage{occuperCol(Y, M, M2)}

Étant donné une matrice $M$, \emph{occuperCol} donne une nouvelle
matrice $M2$ où la colonne $Y$ a été mise à $t$ pour indiquer que la
dame attaque cette colonne. Cette relation utilise la relation
\emph{changerCol} afin de changer chaque ligne.


\subsubsection{occuperLig}

\usage{occuperLig(X, N, M, M2)}

Étant donnée une matrice $M$ (et sa dimension $N$), \emph{occuperLig}
donne une nouvelle matrice $M2$ où la ligne $X$ a été mise à $t$ pour
indiquer que la dame attaque cette ligne. Cette relation utilise la
relation \emph{repeter} qui donnera une nouvelle ligne de $N$ $t$.


\subsubsection{occuperDiagDroite}

\usage{occuperDiagDroite(X, Y, N, M, M2)}

Étant donné une matrice $M$, une position $(X,Y)$ et la dimension $N$,
\emph{occuperDiagDroite} produit la matrice $M2$ où les éléments sur
la diagonale de $(X,Y)$ sont mis à $t$ pour indiquer que la dame
attaque cette diagonale. Cette relation utilise \emph{changerCell}
récursivement pour modifier chaque élément de la diagonale. La
récursivité s'arrête lorsque $X$ ou $Y$ atteint la ligne ou colonne
$N$.

Si $(X,Y)$ se trouve sur la diagonale principale, les deux conditions
d'arrêt sont atteintes en même temps et Prolog produira deux réponses.
Ceci aurait pu être évité si l'utilisation de l'opérateur \emph{cut}
avait été permise.


\subsubsection{occuperDiagGauche}

\usage{occuperDiagGauche(X, Y, N, M, M2)}

Étant donné une matrice $M$, une position $(X,Y)$ et la dimension $N$,
\emph{occuperDiagGauche} produit la matrice $M2$ où les éléments sur
la diagonale de $(X,Y)$ sont mis à $t$ pour indiquer que la dame
attaque cette diagonale. Cette relation utilise \emph{changerCell}
récursivement pour modifier chaque élément de la diagonale. La
récursivité s'arrête lorsque $X$ ou $Y$ atteint la ligne $N$ ou
colonne 1.


Si $(X,Y)$ se trouve sur la diagonale secondaire, les deux conditions
d'arrêt sont atteintes en même temps et Prolog produira deux réponses.

\subsubsection{occuper}

\usage{occuper(N, X, Y, M, M2)}

Cette relation fait appel à \emph{occuperCol}, \emph{occuperLig},
\emph{occuperDiagDroite} et \emph{occuperDiagGauche} pour marquer dans
$M2$ toutes les positions attaquées dans $M$ si on place une dame en
position $(X,Y)$.

\subsubsection{nth}

\usage{nth(I, L, T)}

\emph{nth} unifie avec $T$ le $I$ème élément de la liste $L$.

\subsubsection{estLibre}

\usage{estLibre(M, X, Y)}

\emph{estLibre} utilise \emph{nth} deux fois pour obtenir la valeur en
position $(X, Y)$ de la matrice d'attaque $M$. Cette valeur est
ensuite comparée avec $f$. Si l'élément est $f$, alors la case est
libre et sécuritaire, sinon elle ne l'est pas.

\subsubsection{placer}

\usage{placer(X, L, N, M)}

\emph{placer} tente de placer sur chaque ligne $X$ une dame dans une
colonne $Y$ qui se trouvent dans la liste $L$. Si la case est libre
(tel que déterminé par \emph{estLibre}), alors la case, sa colonne, sa
ligne et ses diagonales sont marquées comme étant attaquées et on
tente d'insérer la dame suivante sur la ligne suivante. Si toutes les
dames ont pu être placées, on a un placement valide.


\subsubsection{ndames}

\usage{ndames(N, L)}

Cette relation est le programme principal. Elle peut vérifier si une
liste de positions est un placement valide des dames sur l'échiquier.
Elle peut également générer les positions valides pour un échiquier de
taille $N$.


\section{Impressions de Prolog}

Prolog est intéressant pour générer automatiquement des solutions que
l'on devrait faire manuellement dans des langages impératifs ou
fonctionnels (e.g.: accumuler dans une liste les positions valides
pour le problèmes des $N$-dames).

La restriction de n'utiliser aucune des fonctionnalités avancées
(impures) de Prolog limite considérablement les avenues possibles pour
résoudre les problèmes. Par exemple, plutôt que d'utiliser une
matrice d'attaque, on aurait pu avoir une règle \emph{estSécuritaire}
qui détermine si une case n'est attaquée par aucune dame. Comme il
est trivial de déterminer si deux reines s'attaquent mutuellement, il
serait trivial avec la négation de déterminer si deux reines ne
s'attaquent pas. Mais comme la négation est interdite, cette avenue a
été abandonnée.

\end{document}

