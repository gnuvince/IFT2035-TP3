\documentclass[10pt]{article}

\usepackage[utf8]{inputenc}
\usepackage[french]{babel}
\usepackage{amsmath}
\usepackage{amsfonts}
\usepackage{amssymb}
\usepackage{graphicx}
\usepackage{parskip}

\newcommand{\usage}[1]{\textbf{Utilisation: }\emph{#1}}

\begin{document}

\title{Rapport - Devoir 3}
\date{Décembre 2010}
\author{Vincent Foley-Bourgon (FOLV08078309) \and
    Eric Thivierge (THIE09016601)}

\maketitle

\section{$N$-Reines}

\subsection{Fonctionnement général}

Pour un plateau de taille NxN, le programme \emph{nreines} utilise le fait qu'une solution à ce problème peut prendre la forme d'une permutation des éléments d'une liste [1..N]. Le programme prend l'approche générer-tester pour générer les permutations possibles et tester lesquelles sont une solution au problème des N reines.

Le programme utilise la relation \emph{interval} pour générer la liste de base [1..N], la relation \emph{perm} pour générer les permutations possibles de la liste de base et la relation \emph{secure} pour tester si une permutation donnée est une solution au problème.

\subsection{Description des relations utilisées}

\subsubsection{interval}
\usage{interval(L, U, R)}

Cette relation construit récursivement la liste [L..U] en concatenant [L] et interval((L+1), U, R).

\subsubsection{perm}
\usage{perm([X$|$R], L)}

Cette relation unifie \emph{L} à \emph{[X$|$R]} en utilisant le fait que \emph{L} est une permutation de \emph{R} à laquelle on ajoute \emph{X}.

\subsubsection{secure}
\usage{secure(P1, P2, [R1, R2$|$Rs])}

Cette relation vérifie récursivement si la permutation donnée est une solution au problème des N reines. La relation utilise la relation auxilliaire \emph{secureAux} pour vérifier que:
\begin{itemize}
\item les positions (P1, R1) et (P2, R2) ne s'attaquent pas
\item récursivement, la position (P1, R1) n'attaque pas les éléments restant de la permutation (\emph{Rs}) et
\item récursivement, la position (P2, R2) n'attaque pas les éléments restant de la permutation (\emph{Rs}).
\end{itemize}

\subsubsection{secureAux}
\usage{secureAux((X1, Y1), (X2, Y2))}

Cette relation utilise le fait que deux positions s'attaquent si elles sont sur la même ligne, colonne ou diagonale. Pour vérifier si deux positions sont sur une même diagonale, la relation vérifie si la variation en x (\emph{dX}) et la variation en y (\emph{dY}) entre ces positions ne sont pas égales.

\end{document}

