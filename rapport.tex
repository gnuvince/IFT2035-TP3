\documentclass[10pt]{article}

\usepackage[utf8]{inputenc}
\usepackage[french]{babel}
\usepackage{amsmath}
\usepackage{amsfonts}
\usepackage{amssymb}
\usepackage{graphicx}
\usepackage{parskip}

\newcommand{\usage}[1]{\textbf{Utilisation: }\emph{#1}}

\begin{document}

\title{Rapport - Devoir 3}
\date{Décembre 2010}
\author{Vincent Foley-Bourgon (FOLV08078309) \and
    Eric Thivierge (THIE09016601)}

\maketitle

\section{Informations générales}

Notre programme a été testé sous l'environnement SWI-Prolog.

\section{$N$-Dames}

\subsection{Fonctionnement général}

Pour un plateau de taille $N \times N$, le programme \emph{ndames}
utilise le fait qu'une solution à ce problème peut prendre la forme
d'une permutation des éléments d'une liste $[1..N]$. Le programme prend
l'approche générer-tester pour générer les permutations possibles et
tester lesquelles sont des solutions au problème des $N$-dames.

Le programme utilise la relation \emph{interval} pour générer la liste
de base $[1..N]$, la relation \emph{perm} pour générer les permutations
possibles de la liste de base et la relation \emph{secure} pour tester
si une permutation donnée est une solution au problème.

\subsection{Description des relations utilisées}

\subsubsection{interval}
\usage{interval(L, U, R)}

Cette relation construit récursivement la liste $[L..U]$ en
concaténant L et interval((L+1), U, R).

\subsubsection{perm}
\usage{perm([X$|$R], L)}

Cette relation unifie \emph{L} à \emph{[X$|$R]} en utilisant le fait
que \emph{L} est une permutation de \emph{R} à laquelle on ajoute
\emph{X}.

\subsubsection{secure}
\usage{secure(P1, P2, [R1, R2$|$Rs])}

Cette relation vérifie récursivement si la permutation donnée est une
solution au problème des $N$-dames. La relation utilise la relation
auxiliaire \emph{secureAux} pour vérifier que:
\begin{itemize}
\item les positions (P1, R1) et (P2, R2) ne s'attaquent pas
\item récursivement, la position (P1, R1) n'attaque pas les éléments
  restant de la permutation (\emph{Rs}) et
\item récursivement, la position (P2, R2) n'attaque pas les éléments
  restant de la permutation (\emph{Rs}).
\end{itemize}

\subsubsection{secureAux}
\usage{secureAux((X1, Y1), (X2, Y2))}

Cette relation utilise le fait que deux positions s'attaquent si elles
sont sur la même ligne, colonne ou diagonale. Pour vérifier si deux
positions sont sur une même diagonale, la relation vérifie si la
variation en x (\emph{dX}) et la variation en y (\emph{dY}) entre ces
positions ne sont pas égales.



\section{Labyrinthe}

\subsection{Fonctionnement général}

Étant donné un labyrinthe sans cycle $L$ de dimensions $M \times N$,
la relation \emph{traverser} unifiera $C$ avec l'unique chemin allant
de $[M-1, N-1]$ à $[0, 0]$.


\subsection{Description des relations utilisées}

\subsubsection{elem}

\usage{elem(M, I, J, E)}

Étant donné une matrice $M$ et des coordonnées $I$ et $J$, \emph{elem}
unifie $E$ avec la valeur de $M_{ij}$.

\subsubsection{dimensions}

\usage{dimensions(Lab, [M, N])}

Étant donné un labyrinthe, la relation \emph{dimensions} unifiera $M$
avec le nombre de lignes horizontales et $N$ avec le nombre de
colonnes verticales.


\subsubsection{voisins}

\usage{voisins(M, P1, P2)}

la relation \emph{voisins} détermine si \emph{P1} et \emph{P2} sont
voisins.  Les conditions pour êtres voisins sont les suivantes:
\begin{itemize}
\item Les deux cases doivent contenir la valeur $t$
\item Les deux cases ont une distance verticale de 1 ou (exclusif) une
  distance horizontale de 1.
\end{itemize}


\subsubsection{prodAux}

\usage{prodAux(X, Ys, Zs)}

Une relation auxiliaire utilisée par \emph{prodCartesien},
\emph{prodAux} unifie dans \emph{Zs} le ``cons'' de \emph{X} avec tous
les éléments de \emph{Ys}.

Exemple: \texttt{prodAux(0, [0,1,2,3], R) -> R = [[0,0], [0,1], [0,2],
    [0,3]]}

\subsubsection{prodCartesien}

\usage{prodCartesien(Xs, Ys, Zs)}

\emph{prodCartesien} unifie \emph{Zs} avec le résultat d'appeller
\emph{prodAux} pour tous les éléments de \emph{Xs} avec \emph{Ys},
donnant ainsi le produit cartésien des deux listes.


\subsubsection{chemintInt}

\usage{chemintInt(L, NV, X, Y, C)}

Étant donné un labyrinthe \emph{L}, une liste des cases non-visitées
\emph{NV}, le point de départ \emph{X}, le point d'arrivée \emph{Y},
\emph{chemintInt} unifie récursivement dans \emph{C} le chemin
intermédiaire entre \emph{X} et \emph{Y}.

À chaque itération, les cases visitées sont éliminées de \emph{NV}
pour éviter de revenir sur nos pas.


\subsubsection{traverser}

\usage{traverser(Lab, Chemin)}

Étant donné un labyrinthe \emph{Lab}, \emph{traverser} va utiliser les
relations précédentes pour unifier \emph{Chemin} avec la liste de
positions à suivre pour sortir du labyrinthe.


\section{Impressions de Prolog}

Prolog est intéressant pour générer automatiquement des solutions que
l'on devrait faire manuellement dans des langages impératifs ou
fonctionnels (e.g.: accumuler dans une liste les positions valides
pour le problèmes des $N$-dames).  Ainsi, les solutions aux problèmes
sont surprenamment courtes.

Il est assez difficile de passer au paradigme de programmation
logique. Une fois cette étape passée on peut voir que les problèmes
que nous avions à résoudre ont une solution que l'on peut exprimer de
façon concise et élégante en Prolog. On peut également comprendre que
Prolog peut résoudre certains types de problèmes avec une grande
efficience. Énoncer des relations entre les différents états de
résolution d'un problème s'avère souvent plus difficile que d'écrire
les étapes impératives requises.

La restriction de n'utiliser aucune des fonctionnalités impures de
Prolog a limité les avenues possibles pour résoudre les problèmes.
Par exemple, énoncer une relation d'attaque dans le problèmes des
$N$-dames est trivial, et on peut facilement définir une relation de
sécurité en disant qu'une case est sécure si elle n'est \emph{pas}
attaquée.  Hélas, comme l'utilisation de la négation n'était pas
permise, il a fallu trouver une façon plus astucieuse d'exprimer la
relation de sécurité.


\end{document}

